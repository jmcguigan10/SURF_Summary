\documentclass[10pt]{article}
\usepackage[margin=1in]{geometry}
\usepackage{amsmath,amssymb}
\usepackage{booktabs}
\usepackage{siunitx}
\usepackage[colorlinks=true,linkcolor=black,citecolor=black,urlcolor=black]{hyperref}
\usepackage{enumitem}
\setlist[itemize]{itemsep=0.25em, topsep=0.25em}
\setlength{\parskip}{0.35em}
\setlength{\parindent}{0pt}

\begin{document}

\begin{center}
{\Large \textbf{Summer 2025 Research Summary}}\\[2pt]
{\large \textbf{Machine Learning Flags Fast Neutrino-Flavor Instabilities}}\\[4pt]
John McGuigan \quad Advisor: Sherwood Richers\\
Department of Physics and Astronomy, University of Tennessee, Knoxville\\

\textbf{Abstract.}
Core-collapse supernovae and neutron-star mergers host extreme conditions where neutrino self-interactions can trigger \emph{fast flavor instabilities} (FFI) on cm/ns scales.
I trained heavily regularized multi-layer neural networks (MLNNs; 4--6 layers, hundreds of neurons wide) in PyTorch on $\sim 8\times 10^5$ labeled snapshots spanning seeded-unstable, guaranteed-stable, and neutron-star-merger (NSM) cases.
After a systematic hyperparameter search (dropout, batch normalization, weight decay, batch size, learning rate schedules, and early stopping) and class-weighted training to address the $\sim$5--6\% unstable minority class, the best model reaches $F_1 \approx 0.95$ on held-out data while remaining small enough for in-situ use in simulation codes.
The network outputs a calibrated $P(\mathrm{FFI}\mid x)$ per grid cell, replacing an expensive dispersion solve with negligible overhead in principle; integration tests and wall-time benchmarks are planned for production hydrodynamics runs.

\subsection*{Background \& Physics}
Neutrinos carry away $\gtrsim 99\%$ of the gravitational binding energy in core collapse, and their angular electron-lepton-number (ELN) spectra can admit crossings that seed FFI.
In the mean-field limit, the flavor density matrix $\rho_{\mathbf{p}}$ obeys
\begin{equation*}
i\,\partial_t \rho_{\mathbf{p}} = \left[ H_{\mathrm{vac}} + H_{\mathrm{mat}} + H_{\nu\nu}, \, \rho_{\mathbf{p}} \right],
\end{equation*}
with the self-interaction term
\begin{equation*}
H_{\nu\nu} \propto \mu \!\int (1-\cos\theta)\, G(\mathbf{v})\, d\Omega,
\end{equation*}
where $G(\mathbf{v})$ is the ELN angular distribution and $\mu$ sets the interaction scale. The presence of ELN crossings can drive flavor conversion on microscopic (cm/ns) scales that are numerically prohibitive to resolve directly across all cells in multidimensional simulations.

\subsection*{Data \& Features}
\begin{itemize}
  \item Labeled examples: $\sim 8\times 10^5$ per-cell snapshots (stable vs.\ FFI-unstable), drawn from CCSN and NSM datasets.
  \item Feature vector: 27 scalars per cell (thermodynamic and neutrino-moment quantities including density, $Y_e$, entropy, flux integrals, ELN-crossing proxies).
  \item Class balance: $\approx$5--6\% unstable.
  \item Train/validation split: 80/20 with fixed folds; multiple deterministic seeds used to test reproducibility.
\end{itemize}

\subsection*{Model \& Training}
\begin{itemize}
  \item Architecture: fully-connected MLP (4--6 hidden layers, widths in the few-hundreds), ReLU activations, BatchNorm, dropout.
  \item Loss \& imbalance handling: class-weighted binary cross-entropy with $L_2$ regularization,
  \begin{equation*}
  \mathcal{L} = -w_1\, y\log \hat y - w_0 (1-y)\log(1-\hat y) + \lambda \sum_{\ell}\lVert W_\ell\rVert_2^2,
  \end{equation*}
  optimized with AdamW and a cosine/step learning-rate schedule.
  \item Batching \& schedules: batch sizes $4\,096$--$16\,384$; learning rate $\sim 3\times 10^{-3}$ with decay; early stopping to prevent overfitting.
  \item Thresholding: sweep the classification cutoff on $P(\mathrm{FFI}\mid x)$ to maximize the $F_1$ score (balances precision and recall).
\end{itemize}

\subsection*{Validation \& Reproducibility}
\begin{itemize}
  \item Metrics: precision, recall, $F_1 = \frac{2PR}{P+R}$ on a held-out test set.
  \item Results at the best cutoff are stable across seeds; representative numbers are below.
\end{itemize}

\begin{center}
\begin{tabular}{@{}lccc@{}}
\toprule
Seed & Recall & Precision & $F_1$ \\\midrule
17 & 0.921 & 0.983 & 0.951 \\
43 & 0.919 & 0.974 & 0.945 \\
\bottomrule
\end{tabular}
\end{center}

Median performance across top configurations sits near $F_1 \approx 0.94$--$0.95$ with high precision ($\approx$0.97--0.98) and strong recall ($\approx$0.90--0.92).
I also trained smaller variants (fewer layers and narrower widths) to minimize inference cost for in-situ usage; these retain competitive $F_1$ with modest drops in recall.

\subsection*{Deployment Plan \& Impact}
\begin{itemize}
  \item \textbf{In-situ prediction:} export to TorchScript/ONNX and batch per-cell inference on CPU or GPU within radiation-hydrodynamics codes.
  \item \textbf{Usage:} treat $P(\mathrm{FFI}\mid x)$ as a flag or risk score to (i) enable flavor-aware closures, or (ii) trigger higher-fidelity solvers only where the network predicts instability.
  \item \textbf{Speed-aware design:} architecture constrained for real-time deployability; end-to-end wall-time benchmarks on production grids are the next step.
\end{itemize}

\subsection*{What I Build So Far}
\begin{itemize}
  \item End-to-end PyTorch training pipeline (data loaders, augmentation hooks, logging, and plotting).
  \item Automated hyperparameter searches over depth, width, dropout, weight decay, batch size, and learning-rate schedules with early stopping.
  \item Class-weighted training to handle severe imbalance; threshold sweeps and calibration curves to set operating points.
  \item Reproducibility studies across multiple seeds; convergence diagnostics and overfitting detection.
  \item Compact models for faster inference while maintaining high sensitivity to the minority class.
\end{itemize}

\textbf{Next Steps.}
Runtime profiling inside a test hydrodynamics code; uncertainty calibration (temperature scaling, isotonic regression), out-of-distribution checks for new progenitors/NSM conditions, and active-learning loops that prioritize follow-up dispersion solves where the classifier is uncertain.
\end{center}

\end{document}
